%-----------------------------------------------------------------------------
%
%               Template for sigplanconf LaTeX Class
%
% Name:         sigplanconf-template.tex
%
% Purpose:      A template for sigplanconf.cls, which is a LaTeX 2e class
%               file for SIGPLAN conference proceedings.
%
% Author:       Paul C. Anagnostopoulos
%               Windfall Software
%               978 371-2316for the
%               paul@windfall.com
%
% Created:      15 February 2005
%
%-----------------------------------------------------------------------------


\documentclass{sigplanconf}  %[preprint]

% The following \documentclass options may be useful:
%
% 10pt          To set in 10-point type instead of 9-point.
% 11pt          To set in 11-point type instead of 9-point.
% authoryear    To obtain author/year citation style instead of numeric.

\usepackage{amsmath}
\usepackage{amssymb}
\usepackage{amsthm}

\usepackage{multicol}
\usepackage{multirow}
\usepackage{graphicx}
\usepackage{color}
\usepackage{subfigure}
\usepackage{epsfig}
\usepackage{fancybox}
\usepackage{fancyvrb}

\usepackage{amssymb}
\usepackage{amsthm}

\usepackage{hyperref} % comes last, as it redefines a couple of commands
\hypersetup{
    colorlinks,%
    citecolor=black,%
    filecolor=black,%
    linkcolor=black,%
    urlcolor=blue
}

\DefineVerbatimEnvironment{colorcode}%
        {Verbatim}{fontsize=\scriptsize,commandchars=\\\{\}}


\definecolor{DikuRed}{RGB}{130,50,32}
\newcommand{\emp}[1]{\textcolor{DikuRed}{ #1}}
\definecolor{CosGreen}{RGB}{10,100,70}
\newcommand{\emphh}[1]{\textcolor{CosGreen}{ #1}}


\newcommand{\mymath}[1]{$ #1 $}
\newcommand{\myindx}[1]{_{#1}}
\newcommand{\myindu}[1]{^{#1}}
\newcommand{\mymathbb}[1]{\mathbb{#1}}

\hyphenation{ho-mo-mor-phism}
\hyphenation{list-ho-mo-mor-phism}
\hyphenation{mul-ti-pli-ca-tion}
\hyphenation{re-le-vant}
\hyphenation{asso-ci-a-ti-vi-ty}


%%%%%%%%%%%%%%%%%%%%%%%
% comments
\usepackage{color}
\newcommand{\comment}[2]{\textcolor{red}{\scriptsize \textsf \textbf{#1:}{#2}}}

% to switch comments off, activate this definition
% \renewcommand{\comment}[2]{}

%%%%%%%%%%%%%%%%%%%%%%%

\begin{document}

\conferenceinfo{FHPC'13,} {September 23, 2013, Boston, Massachusets.}
\CopyrightYear{2013}
%\copyrightdata{978-1-4503-1577-7/12/09}


\title{A Structural-Analysis Algorithm for Fusion}
\subtitle{$\mathcal{L}_0$ Status Report}
%\subtitle{Subtitle Text, if any}

%%%%%%%%%%%%%%%%%%%
%%% AUTHORS INF %%%
%%%%%%%%%%%%%%%%%%%

\authorinfo{Troels Henriksen, Cosmin E. Oancea, Fritz Henglein}
           {HIPERFIT, Department of Computer Science, University of Copenhagen (DIKU)}
           {cosmin.oancea@diku.dk, athas@sigkill.dk, henglein@diku.dk}


%%%%%%%%%%%%%%%%%%%
%%%%%%%%%%%%%%%%%%%
%%%%%%%%%%%%%%%%%%%

\maketitle
%\renewcommand{\comment}[2]{}



\begin{abstract}

Fusion is one of the most important code transformations as it 
has the potential to substantially optimize both the memory hierarchy 
time overhead and (sometimes asymptotically) the space requirement.
%
In imperative languages, the legality of loop-fusion is typically 
verified by dependency analysis on arrays applied at loop-nest level.
Such analysis, however, has often been labeled as ``heroic effort''
and, if at all, is supported only in its simplest and most
conservative form in industrial compilers.  

In functional languages, fusion is naturally and more easily derived
as a producer-consumer relation between program constructs that expose
a richer, higher-order algebra of program invariants, 
%from the richer, higher-order semantics of program invariants, 
such as the {\tt map-reduce} list homomorphisms. %algebra.   

Related implementations in the functional context typically 
apply fusion only when the to-be-fused producer is used exactly once,
i.e., in the consumer.   This guarantees that the transformation is
conservative: the resulting program does not duplicate computation.

We show that the above restriction is more conservative than needed,
and present a structural-analysis algorithm, inspired
from the {\tt T$_1$-T$_2$} transformation for reducible data flow,
that enables fusion even in some cases when the producer is used 
in different consumers {\em and} without duplicating computation.  

We report an implementation of the fusion algorithm for a 
{\em functional}-core language, named $\mathcal{L}_0$, which is intended 
to support {\em nested} parallelism across {\em regular} 
multi-dimensional arrays.  We succinctly describe $\mathcal{L}_0$'s
semantics and the compiler infrastructure on which the fusion
transformation relies.

\end{abstract}

%\category{CR-number}{subcategory}{third-level}
\category{D.1.3}{Concurrent Programming}{Parallel Programming}
%\category{D.1.3}{Programming Techniques}{Concurrent Programming}%%[Parallel Programming]
\category{D.3.4}{Processors}{Compiler}


\terms
Performance, Design, Algorithms

\keywords
fusion, autoparallelization, functional language

\section{Introduction}
\label{sec:Introduction}

%%%%%%%%%%%%%%%%%%%%%%%%%%%%%%%%%%%%%%%%%%%%%%%%%%%%%%%%%%%%%%
%%% HL Introduction: financial motivation + GPU motivation %%%
%%% + our generic-pricing case study + short description   %%%
%%%%%%%%%%%%%%%%%%%%%%%%%%%%%%%%%%%%%%%%%%%%%%%%%%%%%%%%%%%%%%

%The motivation for the $\mathcal{L}_0$ language steams from
%one of the goals of the {\sc hiperfit} project

One of the main goals of the {\sc hiperfit} project has been to
develop the infrastructure necessary to write real-world, big-data 
financial applications in a hardware-independent language that can 
be efficiently executed on massively parallel hardware, e.g., {\sc gpgpu}.  % various 

In this sense we have examined several such computational kernels~\cite{PricingFHPC}, 
originally implemented in languages such as {\tt OCamel, Python, C++, C}  
and measuring in the range of hundreds/thousands lines of compact code, 
with two main objectives in mind: 
\begin{itemize}
    \item[1.] What should be a suitable core language that, on the
                one hand, would allow a relatively straight-forward 
                code translation, and, on the other hand, would
                preserve the algorithmic invariants that are needed
                to optimize the application globally?
    \item[2.] What compiler optimizations would result in efficiencies
                comparable to the hardware hand-tuned version 
                of the code?
\end{itemize}

The answer to the first question has been %, not surprisingly, 
a {\em functional} language, dubbed $\mathcal{L}_0$, supporting %with support for 
\texttt{map-reduce} {\em nested} parallelism on {\em regular} arrays, i.e., 
the size of each dimension is constant at runtime:\\
It is {\em functional} because we would rather invest compiler effort
in exploiting high-level program invariants rather than in proving them.
The common example in this sense is parallelism: {\tt map-reduce} 
constructs are inherently parallel, while Fortran-style \texttt{do} 
loops require sophisticated analysis to decide parallelism. 
Furthermore, such techniques~\cite{Blume94RangeTest,SUIF,CosPLDI} %there is solid evidence that such 
have not yet been integrated in the repertoire of commercial compilers,
likely due to ``heroic effort'' concerns, albeit
 (i) their effectiveness was demonstrated on comprehensive suites, and
(ii) some of them were developed more than a decade ago.
%
It is {\em nested} because our suite exhibits several layers of 
parallelism that cannot be exploited by flat parallelism in the style of 
{\sc repa}~\cite{REPA}, e.g., several innermost {\tt scan} or 
{\tt reduce} operations and at least one (semantically)
sequential loop per benchmark.  
%
Finally, it is {\em regular} because our suite does not require irregular 
arrays in the sense of {\sc nesl}~\cite{BlellochCACM96NESL}, and 
regular arrays are more amenable to compiler optimizations.

Perhaps less expectedly, the answer to the second question seems to be 
that a common ground needs to be found between functional and imperative
optimizations and, to a less extent,  between language constructs.
Much in the same way in which (data) parallelism seems to be generated by
a combination of {\tt map}, {\tt reduce}, and {\tt scan} operations, 
the optimization opportunities, e.g., enhancing the degree of parallelism 
and reducing  the memory time and space overheads, seem solvable via a 
combination of {\tt fusion}, {\tt transposition}, loop {\tt interchange} 
and loop {\tt distribution}.
%
It follows that loops are necessary in the intermediate representation,
regardless of whether they are provided as a language construct or are
derived from tail-recursive functions. 

Finally, an indirect consequence of having to deal with sequential (dependent) 
%, which semantically update an unknown array index at a time, 
loops is that $\mathcal{L}_0$ provides support for ``in-place updates'' of 
array elements. The semantics is the functional one, i.e., %the result is a
deep copy of the original array but with the corresponding element replaced,
\texttt{intersected} with the imperative one, i.e., if aliasing may prevent 
an in-place implementation a compile-time error is signaled.   The approach
enables a (optimized) cost model that the user likely assumes, while 
preserving the functional semantics. 
%allows both a functional semantics and the cost model that the user likely assumes.

Section~\ref{sec:Prelim} provides a brief overview of the $\mathcal{L}_0$ language
and of the enabling optimizations for fusion.

%\footnote{
%The reasons are that our current benchmark does not require irregular arrays in 
%the sense of {\sc nesl}~\cite{BlellochCACM96NESL}, and regular arrays are more
%amenable to compiler optimizations.
%}    
%L$0$ is a first-order functional core-language intended to support nested 
%parallelism across regular multi-dimensional arrays, 
%%%, i.e., sizes of each array dimension match,
%by means of the typical set of second-order array combinators such
%as \texttt{map}, \texttt{reduce}, \texttt{scan}, \texttt{filter}, \texttt{zip}, etc.





%In the remainder of this section we provide a rationale for our case study and an overview
%of the optimization techniques evaluated.  In the following sections we present the functional
%formulation of the pricing algorithm (Section~\ref{sec:AlgLang}), the optimizations for
%compiling it to {\tt OpenCL} (Section~\ref{sec:Optimizations}), the empirical evaluation of the
%optimizations' impact (Section~\ref{sec:ExpRes}),
%a review of related work on imperative and functional parallelization (Section~\ref{sec:RelWork}),
%and finally our conclusions as to what has been accomplished so far and which future work this suggests
%(Section~\ref{sec:Concl}).

\section{Preliminaries: $\mathcal{L}_0$ and Enabling Optimizations}
\label{sec:Prelim}

For Troels:
\begin{itemize}
    \item[1.] Figure with language grammar or \textsc{AbSyn} $+$ brief explanations.
    \item[2.] One or more seducing code examples with loops and in-place updates (may I suggest TRIDAG?) 
                $+$ brief explanation of loop semantics (equivalence to tail-recursive function) $+$
    \item[3.] in-place update semantics and checking (uniqueness types, aliasing analysis)
    \item[4.] Figure with enabling optimizations $+$ brief explanation for each.
    \item[5.] Demonstration of how code looks like after tuple-of-array transformation,
                i.e., that would be the input for fusion.
\end{itemize}

\section{Fusion: Motivation and Intuitive Solution}
\label{sec:Intuition}

\section{Fusing Once}
\label{sec:FusingOnce}

\section{Fusion Structural-Analysis Algorithm}
\label{sec:FusionPrg}

\section{Limitations and Possible Extensions}
\label{sec:Discuss}


\section{Related Work}
\label{sec:RelWork}

Discuss fusion in REPA, DPH, Accelerate, Haskell, Obsidian, etc.

\section{Conclusions and Future Work}
\label{sec:Concl}



%%%%%%%%%%%%%%%%%%%%%%%%
%%% END MAIN ARTICLE %%%
%%%%%%%%%%%%%%%%%%%%%%%%

%\appendix
%\section{Appendix Title}
%
%This is the text of the appendix, if you need one.


%%%%%%%%%%%%%%%%%%%%%%%%%%%%%%%%%%%%%%%%%%%%%%%%%%%%%%%%%%%%%%%%%%%

\enlargethispage{\baselineskip}
%\vspace*{-1ex}
\acks
%\vspace*{-1ex}
This research has been partially supported by the Danish
Strategic Research Council, Program Committee for Strategic Growth
Technologies, for the research center 'HIPERFIT: Functional High
Performance Computing for Financial Information Technology'
(\url{http://hiperfit.dk}) under contract number 10-092299.


% We recommend abbrvnat bibliography style.

\bibliographystyle{abbrvnat}
\softraggedright
%\bibliography{FHPC12}

\begin{thebibliography}{49}
\providecommand{\natexlab}[1]{#1}
\providecommand{\url}[1]{\texttt{#1}}
\expandafter\ifx\csname urlstyle\endcsname\relax
  \providecommand{\doi}[1]{doi: #1}\else
  \providecommand{\doi}{doi: \begingroup \urlstyle{rm}\Url}\fi

\bibitem[Allen and Kennedy(2002)]{OptCompModernArch}
R.~Allen and K.~Kennedy.
\newblock \emph{{O}ptimizing {C}ompilers for {M}odern {A}rchitectures}.
\newblock Morgan Kaufmann, 2002.
\newblock ISBN 1-55860-286-0.

\bibitem[Amini et~al.(2011)Amini, Coelho, Irigoin, and
  Keryell]{IrigoinHostAccelOptim}
M.~Amini, F.~Coelho, F.~Irigoin, and R.~Keryell.
\newblock Static {C}ompilation {A}nalysis for {H}ost-{A}ccelerator
  {C}ommunication {O}ptimization.
\newblock In \emph{Int. Work. Lang. and Compilers for Par. Computing (LCPC)},
  2011.

\bibitem[Armstrong and Eigenmann(2008)]{GamessDif}
B.~Armstrong and R.~Eigenmann.
\newblock Application of {A}utomatic {P}arallelization to {M}odern {C}hallenges
  of {S}cientific {C}omputing {I}ndustries.
\newblock In \emph{Int. Conf. Parallel Proc. (ICPP)}, pages 279--286, 2008.

\bibitem[Augustsson et~al.(2008)Augustsson, Mansell, and
  Sittampalam]{Augustsson08Paradise}
L.~Augustsson, H.~Mansell, and G.~Sittampalam.
\newblock Paradise: {A} {T}wo-{S}tage {DSL} {E}mbedded in {Haskell}.
\newblock In \emph{Int. Conf. on Funct. Prog. (ICFP)}, pages 225--228, 2008.

\bibitem[Baskaran et~al.(2010)Baskaran, Ramanujam, and
  Sadayappan]{SadayappanCtoCUDA}
M.~M. Baskaran, J.~Ramanujam, and P.~Sadayappan.
\newblock Automatic {C}-to-{CUDA} {C}ode {G}eneration for {A}ffine {P}rograms.
\newblock In \emph{Int. Conf. on Compiler Construction (CC)}, pages 244--263,
  2010.

\bibitem[Bird(1987)]{BirdListTh}
R.~S. Bird.
\newblock An {I}ntroduction to the {T}heory of {L}ists.
\newblock In \emph{NATO Inst. on Logic of Progr. and Calculi of Discrete
  Design}, pages 5--42, 1987.

\bibitem[Black and Scholes(1973)]{black1973pricing}
F.~Black and M.~Scholes.
\newblock The {P}ricing of {O}ptions and {C}orporate {L}iabilities.
\newblock \emph{The Journal of Political Economy}, pages 637--654, 1973.

\bibitem[Blelloch(1996)]{BlellochCACM96NESL}
G.~Blelloch.
\newblock Programming {P}arallel {A}lgorithms.
\newblock \emph{Communications of the {ACM} (CACM)}, 39\penalty0 (3):\penalty0
  85--97, 1996.

\bibitem[Blume and Eigenmann(1994)]{Blume94RangeTest}
W.~Blume and R.~Eigenmann.
\newblock The {R}ange {T}est: {A} {D}ependence {T}est for {S}ymbolic,
  {N}on-{L}inear {E}xpressions.
\newblock In \emph{Procs. Int. Conf. on Supercomp}, pages 528--537, 1994.

\bibitem[Bratley and Fox(1988)]{Sobol}
P.~Bratley and B.~L. Fox.
\newblock Algorithm 659 {I}mplementing {S}obol's {Q}uasirandom {S}equence
  {G}enerator.
\newblock \emph{ACM Trans. on Math. Software (TOMS)}, 14(1):\penalty0 88--100,
  1988.

\bibitem[Chakravarty et~al.(2007)Chakravarty, Leshchinskiy, Jones, Keller, and
  Marlow]{Chak06DPH}
M.~Chakravarty, R.~Leshchinskiy, S.~P. Jones, G.~Keller, and S.~Marlow.
\newblock Data parallel {H}askell: A status report.
\newblock In \emph{Int. Work. on Declarative Aspects of Multicore Prog.
  (DAMP)}, pages 10--18, 2007.

\bibitem[Chakravarty et~al.(2011)Chakravarty, Keller, Lee, McDonell, and
  Grover]{ArrayAccelerate}
M.~M. Chakravarty, G.~Keller, S.~Lee, T.~L. McDonell, and V.~Grover.
\newblock Accelerating {H}askell {A}rray {C}odes with {M}ulticore {GPUs}.
\newblock In \emph{Int. Work. on Declarative Aspects of Multicore Prog.
  (DAMP)}, pages 3--14, 2011.

\bibitem[Cole(1993)]{ColeNearHom}
M.~Cole.
\newblock Parallel {P}rogramming, {L}ist {H}omomorphisms and the {M}aximum
  {S}egment {S}um {P}roblem.
\newblock In \emph{Procs. of Parco 93}, 1993.

\bibitem[Dang et~al.(2002)Dang, Yu, and Rauchwerger]{R-LRPD}
F.~Dang, H.~Yu, and L.~Rauchwerger.
\newblock The {R-LRPD} {T}est: {S}peculative {P}arallelization of {P}artially
  {P}arallel {L}oops.
\newblock In \emph{Int. Par. and Distr. Processing Symp. (PDPS)}, pages 20--29,
  2002.

\bibitem[Dubach et~al.(2012)Dubach, Cheng, Rabbah, Bacon, and Fink]{Lime}
C.~Dubach, P.~Cheng, R.~Rabbah, D.~F. Bacon, and S.~J. Fink.
\newblock Compiling a {H}igh-{L}evel {L}anguage for {GPU}s.
\newblock In \emph{Int. Conf. Prg. Lang. Design and Implem. (PLDI)}, pages
  1--12, 2012.

\bibitem[Feautrier(1991)]{FeautrierDataflow}
P.~Feautrier.
\newblock {D}ataflow {A}nalysis of {A}rray and {S}calar {R}eferences.
\newblock \emph{Int. Journal of Par. Prog}, 20(1):\penalty0 23--54, 1991.

\bibitem[Gibbons(1996)]{ThirdLHTh}
J.~Gibbons.
\newblock The {T}hird {H}omomorphism {T}heorem.
\newblock \emph{Journal of Functional Programming (JFP)}, 6\penalty0
  (4):\penalty0 657--665, 1996.

\bibitem[Glasserman(2004)]{glasserman2004monte}
P.~Glasserman.
\newblock \emph{Monte Carlo {M}ethods in {F}inancial {E}ngineering}.
\newblock Springer, New York, 2004.
\newblock ISBN 0387004513.

\bibitem[Gorlatch(1996{\natexlab{a}})]{Gorlatch:AntiUnif}
S.~Gorlatch.
\newblock Systematic {E}xtraction and {I}mplementation of
  {D}ivide-and-{C}onquer {P}arallelism.
\newblock In \emph{PLILP'96}, pages 274--288, 1996{\natexlab{a}}.

\bibitem[Gorlatch(1996{\natexlab{b}})]{GorlatchDistrHom}
S.~Gorlatch.
\newblock Systematic {E}fficient {P}arallelization of {S}can and {O}ther {L}ist
  {H}omomorphisms.
\newblock In \emph{Ann. European Conf. on Par. Proc. LNCS 1124}, pages
  401--408. Springer-Verlag, 1996{\natexlab{b}}.

\bibitem[Hall et~al.(2005)Hall, Amarasinghe, Murphy, Liao, and Lam]{SUIF}
M.~W. Hall, S.~P. Amarasinghe, B.~R. Murphy, S.-W. Liao, and M.~S. Lam.
\newblock Interprocedural {P}arallelization {A}nalysis in {SUIF}.
\newblock \emph{Trans. on Prog. Lang. and Sys. (TOPLAS)}, 27(4):\penalty0
  662--731, 2005.

\bibitem[Hammond and Michaelson(2000)]{HammondMichaelson00}
K.~Hammond and G.~Michaelson, editors.
\newblock \emph{{Research Directions in Parallel Functional Programming}}.
\newblock Springer, London, 2000.

\bibitem[Hu et~al.(1999)Hu, Takeichi, and Iwasaki]{HuDiffusion}
Z.~Hu, M.~Takeichi, and H.~Iwasaki.
\newblock Diffusion: {C}alculating {E}fficient {P}arallel {P}rograms.
\newblock In \emph{Int. Work. Partial Eval. and Semantics-Based Prg. Manip.
  (PEPM)}, pages 85--94, 1999.

\bibitem[Hughes(1989)]{Hughes89Why}
J.~Hughes.
\newblock Why {F}unctional {P}rogramming {M}atters.
\newblock \emph{The Computer Journal}, 32\penalty0 (2):\penalty0 98--107, 1989.

\bibitem[Hull(2009)]{hull2009options}
J.~Hull.
\newblock \emph{Options, {F}utures {A}nd {O}ther {D}erivatives}.
\newblock Prentice Hall, 2009.

\bibitem[Joshi(2010)]{joshi2010graphical}
M.~Joshi.
\newblock Graphical {A}sian {O}ptions.
\newblock \emph{Wilmott J.}, 2\penalty0 (2):\penalty0 97--107, 2010.

\bibitem[Lee et~al.(2010)Lee, Yau, Giles, Doucet, and
  Holmes]{GilesLee2010utility}
A.~Lee, C.~Yau, M.~Giles, A.~Doucet, and C.~Holmes.
\newblock On the {U}tility of {G}raphics {C}ards to {P}erform {M}assively
  {P}arallel {S}imulation of {A}dvanced {M}onte {C}arlo {M}ethods.
\newblock \emph{J. Comp. Graph. Stat}, 19\penalty0 (4):\penalty0 769--789,
  2010.

\bibitem[Lee et~al.(2009)Lee, Min, and Eigenmann]{EigenmannOpenMPtoGPU}
S.~Lee, S.-J. Min, and R.~Eigenmann.
\newblock Open{MP} to {GPGPU}: a {C}ompiler {F}ramework for {A}utomatic
  {T}ranslation and {O}ptimization.
\newblock In \emph{Int. Symp. Princ. and Practice of Par. Prog. (PPoPP)}, pages
  101--110, 2009.

\bibitem[Lin and Padua(2000)]{PaduaStackArr}
Y.~Lin and D.~Padua.
\newblock Analysis of {I}rregular {S}ingle-{I}ndexed {A}rrays and its
  {A}pplications in {C}ompiler {O}ptimizations.
\newblock In \emph{Procs. Int. Conf. on Compiler Construction}, pages 202--218,
  2000.

\bibitem[Loogen et~al.(2005)Loogen, Ortega-Mall\'{e}n, and
  {Pe\~{n}a-Mar\'{\i}}]{Eden}
R.~Loogen, Y.~Ortega-Mall\'{e}n, and R.~{Pe\~{n}a-Mar\'{\i}}.
\newblock {Parallel Functional Programming in Eden}.
\newblock \emph{J. of Funct. Prog. (JFP)}, 15\penalty0 (3):\penalty0 431--475,
  2005.

\bibitem[Lu and Mellor-Crummey.(1998)]{ExtRed}
B.~Lu and J.~Mellor-Crummey.
\newblock Compiler {O}ptimization of {I}mplicit {R}eductions for {D}istributed
  {M}emory {M}ultiprocessors.
\newblock In \emph{Int. Par. Proc. Symp. (IPPS)}, 1998.

\bibitem[Mainland and Morrisett(2010)]{Nikola}
G.~Mainland and G.~Morrisett.
\newblock {Nikola: Embedding Compiled {GPU} Functions in {H}askell}.
\newblock In \emph{Int. Symp. on Haskell}, pages 67--78, 2010.

\bibitem[Marlow et~al.(2011)Marlow, Newton, and
  Peyton~Jones]{MarlowHS11ParMonad}
S.~Marlow, R.~Newton, and S.~Peyton~Jones.
\newblock {A Monad for Deterministic Parallelism}.
\newblock In \emph{Int. Symp. on Haskell}, pages 71--82, 2011.

\bibitem[Moon and Hall(1999)]{Moon99PredArrDataFlow}
S.~Moon and M.~W. Hall.
\newblock Evaluation of {P}redicated {A}rray {D}ata-{F}low {A}nalysis for
  {A}utomatic {P}arallelization.
\newblock In \emph{Int. Symp. Princ. and Practice of Par. Prog. (PPoPP)}, pages
  84--95, 1999.

\bibitem[Morita et~al.(2007)Morita, Morihata, Matsuzaki, Hu, and
  Takeichi]{MoritaWeakInv}
K.~Morita, A.~Morihata, K.~Matsuzaki, Z.~Hu, and M.~Takeichi.
\newblock Automatic {I}nversion {G}enerates {D}ivide-and-{C}onquer {P}arallel
  {P}rograms.
\newblock In \emph{Int. Conf. Prog. Lang. Design and Impl. (PLDI)}, pages
  146--155, 2007.

\bibitem[Nord and Laure(2011)]{NordParCo11}
F.~Nord and E.~Laure.
\newblock Monte {C}arlo {O}ption {P}ricing with {G}raphics {P}rocessing
  {U}nits.
\newblock In \emph{Int. Conf. ParCo}, 2011.

\bibitem[Oancea and Rauchwerger(2012)]{CosPLDI}
C.~E. Oancea and L.~Rauchwerger.
\newblock Logical {I}nference {T}echniques for {L}oop {P}arallelization.
\newblock In \emph{Int. Conf. Prog. Lang. Design and Impl. (PLDI)}, 2012.

\bibitem[Oancea et~al.(2009)Oancea, Mycroft, and Harris]{OanceaSpLIP}
C.~E. Oancea, A.~Mycroft, and T.~Harris.
\newblock A {L}ightweight, {I}n-{P}lace {M}odel for {S}oftware {T}hread-{L}evel
  {S}peculation.
\newblock In \emph{Int. Symp. on Par. Alg. Arch. (SPAA)}, pages 223--232, 2009.

\bibitem[Paek et~al.(2002)Paek, Hoeflinger, and Padua]{LMAD}
Y.~Paek, J.~Hoeflinger, and D.~Padua.
\newblock Efficient and {P}recise {A}rray {A}ccess {A}nalysis.
\newblock \emph{Trans. on Prog. Lang. and Sys. (TOPLAS)}, 24(1):\penalty0
  65--109, 2002.

\bibitem[{Peyton Jones} et~al.(2000){Peyton Jones}, Eber, and Seward]{SPJ00}
S.~{Peyton Jones}, J.-M. Eber, and J.~Seward.
\newblock Composing {C}ontracts: an {A}dventure in {F}inancial {E}ngineering
  (functional pearl).
\newblock In \emph{Int. Conf. on Funct. Prog. (ICFP)}, pages 280--292, 2000.

\bibitem[Pouchet and et~al.(2012)]{PolyhedralOpt}
L.~Pouchet and et~al.
\newblock Loop {T}ransformations: {C}onvexity, {P}runing and {O}ptimization.
\newblock In \emph{Int. Conf. Princ. of Prog. Lang. (POPL)}, 2012.

\bibitem[Pugh and Wonnacott(1998)]{Pugh98NonlinPresb}
W.~Pugh and D.~Wonnacott.
\newblock Constraint-{B}ased {A}rray {D}ependence {A}nalysis.
\newblock \emph{Trans. on Prog. Lang. and Sys.}, 20(3):\penalty0 635--678,
  1998.

\bibitem[Rus et~al.(2003)Rus, Hoeflinger, and Rauchwerger]{HybAn}
S.~Rus, J.~Hoeflinger, and L.~Rauchwerger.
\newblock Hybrid {A}nalysis: {S}tatic \& {D}ynamic {M}emory {R}eference
  {A}nalysis.
\newblock \emph{Int. Journal of Par. Prog}, 31(3):\penalty0 251--283, 2003.

\bibitem[Ryoo et~al.(2008)Ryoo, Rodrigues, Baghsorkhi, Stone, Kirk, and
  Hwu]{PrinciplesMemGPU}
S.~Ryoo, C.~I. Rodrigues, S.~S. Baghsorkhi, S.~S. Stone, D.~B. Kirk, and
  W.-m.~W. Hwu.
\newblock Optimization {P}rinciples and {A}pplication {P}erformance
  {E}valuation of a {M}ultithreaded {GPU} {U}sing {CUDA}.
\newblock In \emph{Int. Symp. Princ. and Practice of Par. Prog. (PPoPP)}, pages
  73--82, 2008.

\bibitem[Strout(2003)]{StroutPhD}
M.~M. Strout.
\newblock \emph{Performance transformations for irregular applications}.
\newblock PhD thesis, 2003.
\newblock AAI3094622.

\bibitem[Trinder et~al.(1996)Trinder, Hammond, {Mattson Jr.}, Partridge, and
  {Peyton Jones}]{GpH}
P.~Trinder, K.~Hammond, J.~{Mattson Jr.}, A.~Partridge, and S.~{Peyton Jones}.
\newblock {GUM: a Portable Parallel Implementation of Haskell}.
\newblock In \emph{Int. Conf. Prg. Lang. Design and Implem. (PLDI)}, pages
  78--88, 1996.

\bibitem[Ueng et~al.(2008)Ueng, Lathara, Baghsorkhi, and Hwu]{CudaLite}
S.-Z. Ueng, M.~Lathara, S.~S. Baghsorkhi, and W.-M.~W. Hwu.
\newblock {CUDA}-{L}ite: {R}educing {GPU} {P}rogramming {C}omplexity.
\newblock In \emph{Int. Work. Lang. and Compilers for Par. Computing (LCPC)},
  pages 1--15, 2008.

\bibitem[Wichura(1988)]{wichura1988algorithm}
M.~Wichura.
\newblock Algorithm {AS} 241: {T}he percentage points of the {N}ormal
  distribution.
\newblock \emph{Journal of the Royal Statistical Society. Series C (Applied
  Statistics)}, 37\penalty0 (3):\penalty0 477--484, 1988.

\bibitem[Yang et~al.(2010)Yang, Xiang, Kong, and Zhou]{YangMemOptim}
Y.~Yang, P.~Xiang, J.~Kong, and H.~Zhou.
\newblock A {GPGPU} {C}ompiler for {M}emory {O}ptimization and {P}arallelism
  {M}anagement.
\newblock In \emph{Int. Conf. Prog. Lang. Design and Implem. (PLDI)}, pages
  86--97, 2010.

\end{thebibliography}

\end{document}
